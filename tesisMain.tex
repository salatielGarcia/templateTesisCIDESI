% =================================================================
%                   Template de tesis para CIDESI
%                   Elaborado por: Salatiel García
% =================================================================
\documentclass[12pt,twoside]{report} % Agregar opción 'draft' para reducir tiempo de compilación
\usepackage[T1]{fontenc}
\usepackage[utf8]{inputenc}
\usepackage[a4paper,centering,margin=2cm]{geometry}%agregar showframe, para mostrar margenes
\usepackage{libertinus}
% \renewcommand*\ttdefault{lm}
\usepackage[]{nimbusmononarrow}
% \renewcommand{\sfdefault}{phv}
% \renewcommand{\rmdefault}{ptm}

\usepackage[chapter,newfloat]{minted}
\setminted{
	autogobble,
	baselinestretch=0.9,
	encoding=utf8,
	fontsize=\footnotesize,
	python3=true,
	obeytabs=false,
	tabsize=4,
	style=default,
	xleftmargin=0mm,
	bgcolor=gray!5,
}
\usepackage[bf]{caption}
\SetupFloatingEnvironment{listing}{name=Código}
\usepackage[hyperref=true,
            url=false,
            isbn=false,
            backref=true, 
			backrefstyle=all+,
            style=ieee,
			dashed=false,
            % citereset=chapter,
            maxcitenames=3,
            maxbibnames=100,
            block=space,]{biblatex}
\addbibresource{tesisMain.bib}
\usepackage{tocloft}
% \cftsetindents{section}{1.5cm}{1.5cm}
% \cftsetindents{subsection}{1.5cm}{1.5cm}
% \cftsetindents{subsubsection}{1.5cm}{1.5cm}
% \cftsetindents{paragraph}{1.5cm}{1.5cm}

\usepackage[spanish,activeacute,es-lcroman,es-noquoting,es-tabla]{babel}
\decimalpoint
\usepackage{hyphenat}
\usepackage{enumitem}
\hyphenation{Tran-sac-tions The-o-ry ex-pe-ri-men-ta-ción}
\usepackage[dvipsnames]{xcolor}
\usepackage[pdftex]{graphicx}
\usepackage{titling}
\usepackage{fancyhdr}
\usepackage{csquotes}
\usepackage{amsmath,amssymb,bm}
\usepackage{steinmetz} %For the angle notation
\usepackage{xfp}
\usepackage{siunitx}
\sisetup{scientific-notation = engineering,parse-numbers=true,per-mode = symbol-or-fraction,exponent-to-prefix, round-mode=places,round-precision=2,fixed-exponent=3,zero-decimal-to-integer=true}
\usepackage{booktabs}
\usepackage{subfig}
\usepackage{tikz}
\usetikzlibrary{shapes.geometric,arrows,fit}
\tikzset{lastxy/.code n args={2}{\pgfgetlastxy{#1}{#2}}}
\usepackage[american,siunitx,RPvoltages]{circuitikz}
\usepackage{pgfplots}
\tikzstyle{arrow} = [-{Stealth[length=5mm, width=3mm]}]
\usepgfplotslibrary{units}
\pgfplotsset{compat=newest,
		legend pos=outer north east,
		y axis style/.style={
			yticklabel style=#1,
			ylabel style=#1,
			y axis line style=#1,
			ytick style=#1
			},
		every tick label/.append style={font=\small},
		minor grid style={gray!15},
		grid style={gray!50},
		width=0.55\textwidth,
		height = 0.25\textheight,
		minor tick num=4,
		axis y line=left,
		axis on top=false,
		axis x line=bottom,
		grid=both,
		legend style={font=\footnotesize},
		label style={font=\small},
		use units,
		unit markings=parenthesis,
		unit code/.code 2 args={\si{#1#2}},
		}
\usepackage{pdfpages}
\usepackage{titlesec}
\newcommand{\bigrule}{\titlerule[0.5mm]}
\titleformat{\chapter}[display] % cambiamos el formato de los
{\bfseries\Huge} % por defecto se usarán caracteres de tamaño
{% contenido de la etiqueta
\titlerule % línea horizontal
\filleft % texto alineado a la derecha
\Large\chaptertitlename\ % "Capítulo" o "Apéndice" en tamaño
\Large\thechapter} % número de capítulo en tamaño \Large
{0mm} % espacio mínimo entre etiqueta y cuerpo
{\filleft} % texto del cuerpo alineado a la derecha
[\vspace{0.5mm} \bigrule] % después del cuerpo, dejar espacio

\setlength{\parskip}{2.5ex plus 0.2ex minus 0.2ex}
\setlength{\abovedisplayskip}{2.0ex plus 0.2ex minus 0.2ex}
\setlength{\belowdisplayskip}{2.0ex plus 0.2ex minus 0.2ex}
\setlength{\textfloatsep}{3.5ex plus 0.2ex minus 0.2ex}
\setlength{\floatsep}{3.5ex plus 0.2ex minus 0.2ex}
\setlength{\intextsep}{3.5ex plus 0.2ex minus 0.2ex}
\setlength{\belowcaptionskip}{-\baselineskip}      %
\addtolength{\belowcaptionskip}{3.5ex plus 0.2ex minus 0.2ex}             %Define el espacio vertical entre figuras 1.5ex 

\global\hyphenpenalty=1000
\hbadness=5000
\vbadness=1000
\hfuzz=5pt
\vfuzz=5pt
\binoppenalty=9999  % default 700
\relpenalty=9999     % default 500
\emergencystretch=1.5em

\marginparsep      10pt
\marginparwidth    20pt
\marginparpush     25pt

% if things get too close, go ahead and let them touch
\lineskip            0pt
\normallineskip      0pt
\lineskiplimit       0pt
\normallineskiplimit 0pt

% The distance from the lower edge of the text body to the
% footline
\footskip 8ex
\parindent    0.0em        %Identación de parrafos
\linespread{1.4}         %Espaciado entre renglones

\renewcommand{\thesubsubsection}{\Alph{subsubsection}}
\setcounter{secnumdepth}{4}
% Definir corrientes de distintos colores en circuitos
\makeatletter
\ctikzset{current arrow color/.initial=black}% create key
\pgfdeclareshape{currarrow}{
    \anchor{center}{
        \pgfpointorigin
    }
        \anchor{tip}{
        \pgfpointorigin
            \pgf@circ@res@step = \pgf@circ@Rlen
                \divide \pgf@circ@res@step by 16
        \pgf@x  =\pgf@circ@res@step
        }
    \behindforegroundpath{      

        \pgfscope
            \pgf@circ@res@step = \pgf@circ@Rlen
            \divide \pgf@circ@res@step by 16

            \pgfpathmoveto{\pgfpoint{-.7\pgf@circ@res@step}{0pt}}
            \pgfpathlineto{\pgfpoint{-.7\pgf@circ@res@step}{-.8\pgf@circ@res@step}}
            \pgfpathlineto{\pgfpoint{1\pgf@circ@res@step}{0pt}}
            \pgfpathlineto{\pgfpoint{-.7\pgf@circ@res@step}{.8\pgf@circ@res@step}}
            \pgfpathlineto{\pgfpoint{-.7\pgf@circ@res@step}{0pt}}           
            \pgfsetcolor{\pgfkeysvalueof{/tikz/circuitikz/current arrow color}}
            \pgfusepath{draw,fill}

        \endpgfscope
    }
}
\makeatother

\let\origdoublepage\cleardoublepage
\newcommand{\clearemptydoublepage}{
  \clearpage
  {\pagestyle{empty}\origdoublepage}
}

\newcommand{\finalDeSec}[1]{
	\begin{minipage}[c]{0.99\textwidth}
		\[
			\S
		\]
	#1 \hfill
	\vfill 
	\end{minipage}
		}	
	
\providecommand\phantomsection{}
\pdfoptionpdfminorversion 6
\usepackage[pdftex,colorlinks=false]{hyperref}
\hypersetup{
   linktoc=all,
   pdfborder={0 0 0},
   unicode=true,          % non-Latin characters in Acrobat’s bookmarks
   pdftoolbar=false,        % show Acrobat’s toolbar?
   pdfmenubar=false,        % show Acrobat’s menu?
   pdffitwindow=false,     % window fit to page when opened
   pdfstartview={FitH},    % fits the width of the page to the window
   pdftitle={Título chingón},    % title
   pdfauthor={Yo mero},     % author
   % pdfsubject={Subject},   % subject of the document
   % pdfcreator={Creador},   % creator of the document
   % pdfproducer={Productor}, % producer of the document
   pdfnewwindow=true,      % links in new window
}

\title{Título chingón}
\author{M. en C. Yo mero}

\begin{document}
	\pdfbookmark{Portada}{portada}
	\pagestyle{empty}
\newgeometry{top=1.5cm,bottom=1.5cm,left=1.5cm,right=1.5cm}
\pagenumbering{alph}
\begin{titlepage}
	\begin{minipage}[c]{0.18\textwidth}
		\begin{center}
			\includegraphics[width=\textwidth]{./img/cidesi.png}
		\end{center}
	\end{minipage}
	\begin{minipage}[c]{0.78\textwidth}
		\begin{center}
			{\large \bf CENTRO DE INGENIRÍA Y DESARROLLO INDUSTRIAL}\\
			\rule{0.95\textwidth}{0.8mm}
		\end{center}
	\end{minipage}\\
	\mbox{%
	{}\hspace{4mm}\vspace*{2mm} \begin{minipage}[c]{0.18\textwidth}	
	\begin{center}
		\def\lll{0.8\textheight}
		\rule{0.8mm}{\lll}\, \rule{0.4mm}{\lll}\, \rule{0.2mm}{\lll} \\
		\vspace*{2mm}
		\includegraphics[width=\textwidth]{./img/portada.pdf}
	\end{center}
	\end{minipage}}
	\begin{minipage}[c]{0.75\textwidth}
		\begin{center}
			\parskip 3em
			\lineskip 0.5em
			\textbf{\huge \MakeUppercase \thetitle} \par
			\bigskip
			\bigskip
			\bigskip
			\textbf{\LARGE TESIS} \par
			\bigskip
			\bigskip
			{\Large QUE PARA OBTENER EL} \\%
			{\Large GRADO  ACADÉMICO DE} \par
			\medskip
			\Large \textbf{DOCTOR EN CIENCIA Y TECNOLOGÍA} \\%
			\Large \textbf{EN MECATRÓNICA} \par
			\bigskip
			{\Large PRESENTA:} \par
			{\LARGE \textbf{\theauthor}} \par
			\vfill\null
			{\hfill\null {\large Querétaro, Qro. mayo 2020}}
		\end{center}
	\end{minipage}
\end{titlepage}

	\clearemptydoublepage

	\thispagestyle{empty}
	\newgeometry{top=25mm,bottom=25mm,left=32.5mm,right=22.5mm}
	\pagenumbering{Roman}

	\thispagestyle{empty}
	% \includepdf{./img/autorizacion.pdf} % Descomentar y agregar autorización
	\clearemptydoublepage

	\thispagestyle{empty}
	% \includepdf{./img/liberacion.pdf} % Descomentar y agregar liberación
	\clearemptydoublepage

	\thispagestyle{empty}
	\pdfbookmark{Resumen}{resumen}
	\begin{abstract}
	Resumen chido.
\end{abstract}

	\clearemptydoublepage

	\thispagestyle{empty}
	\phantomsection
	\pdfbookmark{Agradecimientos}{thanks}
	\vspace*{\fill}
\begin{center}
	\textbf{Agradecimientos} 
\end{center}

A mi primo Arturo que tanto colaboró ...

\vspace*{\fill}

	\clearemptydoublepage

	\pagestyle{fancy}
	\addtolength{\headheight}{1.4mm}
	\fancyhead[RE,LO]{\it \nouppercase{\leftmark}}
	\fancyhead[RO,LE]{\it \nouppercase{\rightmark}}
	\rhead{  }
	\lhead{  }
	\lfoot{ }
	\rfoot{ }
	\renewcommand{\headrulewidth}{.5pt}
	\renewcommand{\footrulewidth}{.3pt}
	\pagenumbering{roman}
	\pdfbookmark{\contentsname}{toc}

	\fancyhead[RE,LO]{ }
	\fancyhead[RO,LE]{\it \nouppercase{\rightmark}}
	% \addcontentsline{toc}{chapter}{Índice general}
	\tableofcontents
	\clearemptydoublepage

	\fancyhead[RE,LO]{\it \nouppercase{\leftmark}}
	\fancyhead[RO,LE]{\it \nouppercase{\rightmark}}
	\pagenumbering{arabic}

	\chapter{Introducción}
\label{ch:intro}
Resumen del capítulo.

\section{Introducción.}
	\label{sec:intro}
	Se pueden agregar citas \cite{vampiro} o referencias a secciones como la sección \ref{sec:intro}. Para ecuaciones usar la Eq. \eqref{eq:asd}.

	\begin{equation}
		x = \frac{-b \pm \sqrt{b^2 - a\ c}}{2\ a}
		\label{eq:asd}
	\end{equation}


	Las figuras se pueden agregar con el comando \texttt{input} y usar \texttt{scalebox} para definir el tamaño adecuado. Además el paquete \textit{tikz} permite la definición de diagramas a bloques.

	\begin{figure}[!hbt]
		\centering
		\scalebox{0.88}{\tikzstyle{startstop} = [rectangle,thick, rounded corners, minimum width=3cm, minimum height=1cm,text centered, draw=black,inner sep=1pt]
\tikzstyle{process} = [rectangle,thick, rounded corners,minimum width=1cm, minimum height=1cm, text centered, draw=black,inner sep=1pt]
\tikzstyle{decision} = [diamond,thick, rounded corners,minimum width=1cm, minimum height=1cm, text centered, draw=black,inner sep=0pt]
\tikzstyle{io} = [trapezium,thick, rounded corners,trapezium left angle=70, trapezium right angle=110, minimum width=5mm, minimum height=1cm, text centered, draw=black,inner sep=5pt]
\tikzstyle{arrow} = [thick,->,>=stealth]
\tikzstyle{path} = [thick,->,>=stealth]
\linespread{0.9}         %Espaciado entre renglones
\begin{tikzpicture}[node distance=2cm]
	\node (marco) [startstop] {Marco teórico};
	\node (param) [io,below of=marco, text width=3cm] {Selección de parámetros};
	\node (dAnt) [process,below of=param, text width=3cm] {Análisis EM};
	\node (fCoup) [decision,below of=dAnt, aspect=2,text width=3cm,node distance=2.5cm] {Compatibilidad};
	\node (dRec) [process,below of=fCoup, text width=3cm,node distance=2.5cm] {Análisis del rectificador};
	\node (eff) [decision,right of=dRec, text width=3cm, node distance=4.5cm,aspect=2] {Incremento en eficiencia};
	\node (int) [process,right of=eff, text width=2.2cm,node distance=3.8cm] {Integración};
	\node (satis) [decision,above of=int, text width=3cm, node distance=2.5cm,aspect=2] {Satisfactorio};
	\node (anRes) [process,right of=satis, text width=3cm, node distance=4cm,] {Análisis de resultados};
	\node (conc) [startstop,above of=anRes, text width=3cm] {Conclusiones};

	\draw [arrow] (marco) -- (param);
	\draw [arrow] (param) -- (dAnt);
	\draw [arrow] (dAnt) -- (fCoup);
	\draw [arrow] (fCoup) -- (dRec);
	\path (fCoup.south) to node[anchor=west]{Si} (dRec);
	% \draw [arrow] (fCoup) -- (freq);
	% \draw [arrow] (freq) |- (dAnt);
	\draw [arrow] (dRec) -- (eff);
	\draw [arrow] (eff) -- (int);
	\path (eff) to node[anchor=south]{Si} (int);
	\draw [arrow] (int) -- (satis);
	\draw [arrow] (satis) -- (anRes);
	\path (satis) to node[anchor=south]{Si} (anRes);
	\draw [arrow] (anRes) -- (conc);
	\draw [arrow] (fCoup) -- ++(-2.5cm,0) |- (dAnt);
	\path (fCoup.west) to ++(-0.4cm,0) to node[anchor=south]{No} (dAnt.west);
	\draw [arrow] (satis) |- (dAnt);
	\path  (satis.west) to ++(0,2.5cm) to node[anchor=south]{No} (dAnt.east);
	\draw [arrow] (eff) -- ++(0,-2cm) -| (dRec);
	\path (eff.south) to ++(0,-1.8cm) to node[anchor=west]{No} (dRec.south);
\end{tikzpicture}
\linespread{1.3}         %Espaciado entre renglones
% trapezium left angle=70, trapezium right angle=110,
}
		\caption{Metodología de solución del problema.}
		\label{fig:metod}
	\end{figure}


\finalDeSec{Conclusiones de capítulo.}



	\clearemptydoublepage
	\chapter{Antecedentes}
\label{ch:backg}
Resumen de capítulo

\section{Primera sección.}
	\label{sec:recProbl}

\finalDeSec{Conclusión de capítulo.}



	\clearemptydoublepage
	\chapter{Diseño de sistemas WPT.}
\label{ch:proy}
Resumen de capítulo.

\section{Primera sección.}
	\label{sec:asdasd}

\finalDeSec{Conclusiones de capítulo.}



	\clearemptydoublepage
	\chapter{Experimentación}
	\label{sec:exp}
	Resumen de capítulo.

	\section{Primera sección.}
		\label{sec:ExpRect}

\finalDeSec{Conclusiones de sección.}

	\clearemptydoublepage
	\chapter{Conclusiones y trabajo futuro}
\label{ch:conc}
Resumen de capítulo.

	\section{Conclusiones}
		\label{sec:conc}
		Soy bien chingón.

	\section{Trabajo futuro.}
		\label{sec:future}
		Ser más chingón.

\finalDeSec{Comentarios finales.}

	\clearemptydoublepage
	\appendix
	\fancyhead[RE,LO]{ }
	\fancyhead[RO,LE]{\it \nouppercase{\rightmark}}
	\chapter{Apéndice 1}
\label{app:app1}
dasdasd

	\clearemptydoublepage


	\phantomsection
	\addcontentsline{toc}{chapter}{Bibliografía.}
	\linespread{1}         %Espaciado entre renglones
	\setlength\bibitemsep{2.5\itemsep}
	\printbibliography
\end{document}
