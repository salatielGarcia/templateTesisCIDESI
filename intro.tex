\chapter{Introducción}
\label{ch:intro}
Resumen del capítulo.

\section{Introducción.}
	\label{sec:intro}
	Se pueden agregar citas \cite{vampiro} o referencias a secciones como la sección \ref{sec:intro}. Para ecuaciones usar la Eq. \eqref{eq:asd}.

	\begin{equation}
		x = \frac{-b \pm \sqrt{b^2 - a\ c}}{2\ a}
		\label{eq:asd}
	\end{equation}


	Las figuras se pueden agregar con el comando \texttt{input} y usar \texttt{scalebox} para definir el tamaño adecuado. Además el paquete \textit{tikz} permite la definición de diagramas a bloques.

	\begin{figure}[!hbt]
		\centering
		\scalebox{0.88}{\tikzstyle{startstop} = [rectangle,thick, rounded corners, minimum width=3cm, minimum height=1cm,text centered, draw=black,inner sep=1pt]
\tikzstyle{process} = [rectangle,thick, rounded corners,minimum width=1cm, minimum height=1cm, text centered, draw=black,inner sep=1pt]
\tikzstyle{decision} = [diamond,thick, rounded corners,minimum width=1cm, minimum height=1cm, text centered, draw=black,inner sep=0pt]
\tikzstyle{io} = [trapezium,thick, rounded corners,trapezium left angle=70, trapezium right angle=110, minimum width=5mm, minimum height=1cm, text centered, draw=black,inner sep=5pt]
\tikzstyle{arrow} = [thick,->,>=stealth]
\tikzstyle{path} = [thick,->,>=stealth]
\linespread{0.9}         %Espaciado entre renglones
\begin{tikzpicture}[node distance=2cm]
	\node (marco) [startstop] {Marco teórico};
	\node (param) [io,below of=marco, text width=3cm] {Selección de parámetros};
	\node (dAnt) [process,below of=param, text width=3cm] {Análisis EM};
	\node (fCoup) [decision,below of=dAnt, aspect=2,text width=3cm,node distance=2.5cm] {Compatibilidad};
	\node (dRec) [process,below of=fCoup, text width=3cm,node distance=2.5cm] {Análisis del rectificador};
	\node (eff) [decision,right of=dRec, text width=3cm, node distance=4.5cm,aspect=2] {Incremento en eficiencia};
	\node (int) [process,right of=eff, text width=2.2cm,node distance=3.8cm] {Integración};
	\node (satis) [decision,above of=int, text width=3cm, node distance=2.5cm,aspect=2] {Satisfactorio};
	\node (anRes) [process,right of=satis, text width=3cm, node distance=4cm,] {Análisis de resultados};
	\node (conc) [startstop,above of=anRes, text width=3cm] {Conclusiones};

	\draw [arrow] (marco) -- (param);
	\draw [arrow] (param) -- (dAnt);
	\draw [arrow] (dAnt) -- (fCoup);
	\draw [arrow] (fCoup) -- (dRec);
	\path (fCoup.south) to node[anchor=west]{Si} (dRec);
	% \draw [arrow] (fCoup) -- (freq);
	% \draw [arrow] (freq) |- (dAnt);
	\draw [arrow] (dRec) -- (eff);
	\draw [arrow] (eff) -- (int);
	\path (eff) to node[anchor=south]{Si} (int);
	\draw [arrow] (int) -- (satis);
	\draw [arrow] (satis) -- (anRes);
	\path (satis) to node[anchor=south]{Si} (anRes);
	\draw [arrow] (anRes) -- (conc);
	\draw [arrow] (fCoup) -- ++(-2.5cm,0) |- (dAnt);
	\path (fCoup.west) to ++(-0.4cm,0) to node[anchor=south]{No} (dAnt.west);
	\draw [arrow] (satis) |- (dAnt);
	\path  (satis.west) to ++(0,2.5cm) to node[anchor=south]{No} (dAnt.east);
	\draw [arrow] (eff) -- ++(0,-2cm) -| (dRec);
	\path (eff.south) to ++(0,-1.8cm) to node[anchor=west]{No} (dRec.south);
\end{tikzpicture}
\linespread{1.3}         %Espaciado entre renglones
% trapezium left angle=70, trapezium right angle=110,
}
		\caption{Metodología de solución del problema.}
		\label{fig:metod}
	\end{figure}


\finalDeSec{Conclusiones de capítulo.}


